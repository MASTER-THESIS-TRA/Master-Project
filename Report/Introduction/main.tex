%!TEX root = ../main.tex 

%%%%%%%%%%%%%%%%%%%%%%%%%%%%%%%%%%%%%%%%%%%%%%%%%%%%%%%%%%%%%%%%%%%%%%%%%%%%%%%% 
%%% INTRODUCTION
%%%%%%%%%%%%%%%%%%%%%%%%%%%%%%%%%%%%%%%%%%%%%%%%%%%%%%%%%%%%%%%%%%%%%%%%%%%%%%%%

\chapter{Introduction}
%%%%%%%%%%%%%%%%%%%%%%%%%%%%%%%%%%%%%%%%%%%%%%%%%%%%%%%%%%%%%%%%%%%%%%%%%%%%%%%%
%%% What is an introduction?
%%%%%%%%%%%%%%%%%%%%%%%%%%%%%%%%%%%%%%%%%%%%%%%%%%%%%%%%%%%%%%%%%%%%%%%%%%%%%%%%
%Shortly describe the project; about 1-2 pages
%\begin{itemize}
%  \item Set the environment/story of the project,
%  \item build that into a problem leading up to the ``hypothesis''
%  \item give an overview of how you have investigated this hypothesis and point to important work you have done; it can be an advantage to use references to exact sections
%\end{itemize}

%This chapter is intended to give the reader the information about your report and build up and expectation of what to come. You want to make the reader aware of the good things you have made, so he has something to look forward to.

%Before starting with a section, a chapter should include a short description of what is to come and possible what assumptions you have made. Again we want to build up the expectation of the reader.

%%%%%%%%%%%%%%%%%%%%%%%%%%%%%%%%%%%%%%%%%%%%%%%%%%%%%%%%%%%%%%%%%%%%%%%%%%%%%%%%
%%% My text
%%%%%%%%%%%%%%%%%%%%%%%%%%%%%%%%%%%%%%%%%%%%%%%%%%%%%%%%%%%%%%%%%%%%%%%%%%%%%%%%

